\documentclass[11pt,]{article}
\usepackage[osf,sc]{mathpazo}
\usepackage{setspace}
\setstretch{1.05}
\usepackage{amssymb,amsmath}
\usepackage{ifxetex,ifluatex}
\usepackage{fixltx2e} % provides \textsubscript
\ifnum 0\ifxetex 1\fi\ifluatex 1\fi=0 % if pdftex
  \usepackage[T1]{fontenc}
  \usepackage[utf8]{inputenc}
\else % if luatex or xelatex
  \ifxetex
    \usepackage{mathspec}
  \else
    \usepackage{fontspec}
  \fi
  \defaultfontfeatures{Ligatures=TeX,Scale=MatchLowercase}
\fi
% use upquote if available, for straight quotes in verbatim environments
\IfFileExists{upquote.sty}{\usepackage{upquote}}{}
% use microtype if available
\IfFileExists{microtype.sty}{%
\usepackage{microtype}
\UseMicrotypeSet[protrusion]{basicmath} % disable protrusion for tt fonts
}{}
\usepackage{hyperref}
\hypersetup{unicode=true,
            pdftitle={DATA 607---Homework No.~1},
            pdfauthor={Ben Horvath},
            pdfborder={0 0 0},
            breaklinks=true}
\urlstyle{same}  % don't use monospace font for urls
\usepackage{color}
\usepackage{fancyvrb}
\newcommand{\VerbBar}{|}
\newcommand{\VERB}{\Verb[commandchars=\\\{\}]}
\DefineVerbatimEnvironment{Highlighting}{Verbatim}{commandchars=\\\{\}}
% Add ',fontsize=\small' for more characters per line
\usepackage{framed}
\definecolor{shadecolor}{RGB}{248,248,248}
\newenvironment{Shaded}{\begin{snugshade}}{\end{snugshade}}
\newcommand{\AlertTok}[1]{\textcolor[rgb]{0.94,0.16,0.16}{#1}}
\newcommand{\AnnotationTok}[1]{\textcolor[rgb]{0.56,0.35,0.01}{\textbf{\textit{#1}}}}
\newcommand{\AttributeTok}[1]{\textcolor[rgb]{0.77,0.63,0.00}{#1}}
\newcommand{\BaseNTok}[1]{\textcolor[rgb]{0.00,0.00,0.81}{#1}}
\newcommand{\BuiltInTok}[1]{#1}
\newcommand{\CharTok}[1]{\textcolor[rgb]{0.31,0.60,0.02}{#1}}
\newcommand{\CommentTok}[1]{\textcolor[rgb]{0.56,0.35,0.01}{\textit{#1}}}
\newcommand{\CommentVarTok}[1]{\textcolor[rgb]{0.56,0.35,0.01}{\textbf{\textit{#1}}}}
\newcommand{\ConstantTok}[1]{\textcolor[rgb]{0.00,0.00,0.00}{#1}}
\newcommand{\ControlFlowTok}[1]{\textcolor[rgb]{0.13,0.29,0.53}{\textbf{#1}}}
\newcommand{\DataTypeTok}[1]{\textcolor[rgb]{0.13,0.29,0.53}{#1}}
\newcommand{\DecValTok}[1]{\textcolor[rgb]{0.00,0.00,0.81}{#1}}
\newcommand{\DocumentationTok}[1]{\textcolor[rgb]{0.56,0.35,0.01}{\textbf{\textit{#1}}}}
\newcommand{\ErrorTok}[1]{\textcolor[rgb]{0.64,0.00,0.00}{\textbf{#1}}}
\newcommand{\ExtensionTok}[1]{#1}
\newcommand{\FloatTok}[1]{\textcolor[rgb]{0.00,0.00,0.81}{#1}}
\newcommand{\FunctionTok}[1]{\textcolor[rgb]{0.00,0.00,0.00}{#1}}
\newcommand{\ImportTok}[1]{#1}
\newcommand{\InformationTok}[1]{\textcolor[rgb]{0.56,0.35,0.01}{\textbf{\textit{#1}}}}
\newcommand{\KeywordTok}[1]{\textcolor[rgb]{0.13,0.29,0.53}{\textbf{#1}}}
\newcommand{\NormalTok}[1]{#1}
\newcommand{\OperatorTok}[1]{\textcolor[rgb]{0.81,0.36,0.00}{\textbf{#1}}}
\newcommand{\OtherTok}[1]{\textcolor[rgb]{0.56,0.35,0.01}{#1}}
\newcommand{\PreprocessorTok}[1]{\textcolor[rgb]{0.56,0.35,0.01}{\textit{#1}}}
\newcommand{\RegionMarkerTok}[1]{#1}
\newcommand{\SpecialCharTok}[1]{\textcolor[rgb]{0.00,0.00,0.00}{#1}}
\newcommand{\SpecialStringTok}[1]{\textcolor[rgb]{0.31,0.60,0.02}{#1}}
\newcommand{\StringTok}[1]{\textcolor[rgb]{0.31,0.60,0.02}{#1}}
\newcommand{\VariableTok}[1]{\textcolor[rgb]{0.00,0.00,0.00}{#1}}
\newcommand{\VerbatimStringTok}[1]{\textcolor[rgb]{0.31,0.60,0.02}{#1}}
\newcommand{\WarningTok}[1]{\textcolor[rgb]{0.56,0.35,0.01}{\textbf{\textit{#1}}}}
\usepackage{graphicx,grffile}
\makeatletter
\def\maxwidth{\ifdim\Gin@nat@width>\linewidth\linewidth\else\Gin@nat@width\fi}
\def\maxheight{\ifdim\Gin@nat@height>\textheight\textheight\else\Gin@nat@height\fi}
\makeatother
% Scale images if necessary, so that they will not overflow the page
% margins by default, and it is still possible to overwrite the defaults
% using explicit options in \includegraphics[width, height, ...]{}
\setkeys{Gin}{width=\maxwidth,height=\maxheight,keepaspectratio}
\IfFileExists{parskip.sty}{%
\usepackage{parskip}
}{% else
\setlength{\parindent}{0pt}
\setlength{\parskip}{6pt plus 2pt minus 1pt}
}
\setlength{\emergencystretch}{3em}  % prevent overfull lines
\providecommand{\tightlist}{%
  \setlength{\itemsep}{0pt}\setlength{\parskip}{0pt}}
\setcounter{secnumdepth}{0}
% Redefines (sub)paragraphs to behave more like sections
\ifx\paragraph\undefined\else
\let\oldparagraph\paragraph
\renewcommand{\paragraph}[1]{\oldparagraph{#1}\mbox{}}
\fi
\ifx\subparagraph\undefined\else
\let\oldsubparagraph\subparagraph
\renewcommand{\subparagraph}[1]{\oldsubparagraph{#1}\mbox{}}
\fi

%%% Use protect on footnotes to avoid problems with footnotes in titles
\let\rmarkdownfootnote\footnote%
\def\footnote{\protect\rmarkdownfootnote}

%%% Change title format to be more compact
\usepackage{titling}

% Create subtitle command for use in maketitle
\newcommand{\subtitle}[1]{
  \posttitle{
    \begin{center}\large#1\end{center}
    }
}

\setlength{\droptitle}{-2em}

  \title{DATA 607---Homework No.~1}
    \pretitle{\vspace{\droptitle}\centering\huge}
  \posttitle{\par}
    \author{Ben Horvath}
    \preauthor{\centering\large\emph}
  \postauthor{\par}
      \predate{\centering\large\emph}
  \postdate{\par}
    \date{August 28, 2018}

\usepackage{eulervm}

\begin{document}
\maketitle

Load libraries:

\begin{Shaded}
\begin{Highlighting}[]
\KeywordTok{library}\NormalTok{(RCurl)}
\end{Highlighting}
\end{Shaded}

First, let's load the data directly from the source (though a copy is
saved in the \texttt{./data/} directory):

\begin{Shaded}
\begin{Highlighting}[]
\NormalTok{data_url <-}\StringTok{ 'https://archive.ics.uci.edu/ml/machine-learning-databases/mushroom/agaricus-lepiota.data'}
\NormalTok{original <-}\StringTok{ }\KeywordTok{getURL}\NormalTok{(data_url) }
\NormalTok{df <-}\StringTok{ }\KeywordTok{read.csv}\NormalTok{(}\DataTypeTok{text=}\NormalTok{original, }\DataTypeTok{header=}\OtherTok{FALSE}\NormalTok{, }\DataTypeTok{stringsAsFactors=}\OtherTok{FALSE}\NormalTok{)}
\KeywordTok{head}\NormalTok{(df)}
\end{Highlighting}
\end{Shaded}

\begin{verbatim}
##   V1 V2 V3 V4 V5 V6 V7 V8 V9 V10 V11 V12 V13 V14 V15 V16 V17 V18 V19 V20
## 1  p  x  s  n  t  p  f  c  n   k   e   e   s   s   w   w   p   w   o   p
## 2  e  x  s  y  t  a  f  c  b   k   e   c   s   s   w   w   p   w   o   p
## 3  e  b  s  w  t  l  f  c  b   n   e   c   s   s   w   w   p   w   o   p
## 4  p  x  y  w  t  p  f  c  n   n   e   e   s   s   w   w   p   w   o   p
## 5  e  x  s  g  f  n  f  w  b   k   t   e   s   s   w   w   p   w   o   e
## 6  e  x  y  y  t  a  f  c  b   n   e   c   s   s   w   w   p   w   o   p
##   V21 V22 V23
## 1   k   s   u
## 2   n   n   g
## 3   n   n   m
## 4   k   s   u
## 5   n   a   g
## 6   k   n   g
\end{verbatim}

Fill in the column names and subset just a handful:

\begin{Shaded}
\begin{Highlighting}[]
\KeywordTok{colnames}\NormalTok{(df) <-}\StringTok{ }\KeywordTok{c}\NormalTok{(}\StringTok{'poisonous'}\NormalTok{,}
                  \StringTok{'cap_shape'}\NormalTok{, }
                  \StringTok{'cap_surface'}\NormalTok{, }
                  \StringTok{'cap_color'}\NormalTok{, }
                  \StringTok{'bruises'}\NormalTok{, }
                  \StringTok{'odor'}\NormalTok{, }
                  \StringTok{'gill_attachment'}\NormalTok{, }
                  \StringTok{'gill_spacing'}\NormalTok{, }
                  \StringTok{'gill_size'}\NormalTok{, }
                  \StringTok{'gill_color'}\NormalTok{, }
                  \StringTok{'stalk_shape'}\NormalTok{, }
                  \StringTok{'stalk_root'}\NormalTok{, }
                  \StringTok{'stalk_surface_above_ring'}\NormalTok{, }
                  \StringTok{'stalk_surface_below_ring'}\NormalTok{, }
                  \StringTok{'stalk_color_above_ring'}\NormalTok{, }
                  \StringTok{'stalk_color_below_ring'}\NormalTok{, }
                  \StringTok{'veil_type'}\NormalTok{, }
                  \StringTok{'veil_color'}\NormalTok{, }
                  \StringTok{'ring_number'}\NormalTok{, }
                  \StringTok{'ring_type'}\NormalTok{, }
                  \StringTok{'spore_print_color'}\NormalTok{, }
                  \StringTok{'population'}\NormalTok{, }
                  \StringTok{'habitat'}\NormalTok{)}

\NormalTok{cols <-}\StringTok{ }\KeywordTok{c}\NormalTok{(}\StringTok{'poisonous'}\NormalTok{, }\StringTok{'bruises'}\NormalTok{, }\StringTok{'gill_size'}\NormalTok{, }\StringTok{'ring_number'}\NormalTok{)}
\NormalTok{df <-}\StringTok{ }\NormalTok{df[cols]}
\KeywordTok{head}\NormalTok{(df)}
\end{Highlighting}
\end{Shaded}

\begin{verbatim}
##   poisonous bruises gill_size ring_number
## 1         p       t         n           o
## 2         e       t         b           o
## 3         e       t         b           o
## 4         p       t         n           o
## 5         e       f         b           o
## 6         e       t         b           o
\end{verbatim}

The remaining task is the de-abbreviate the data, converting each entry
to a meaningful designation.

One way to do this would be to use many \texttt{gsub()} commands.
However, a custom function that makes multiple substitutions at one go
might make the job a little cleaner and easier to read.

The function \texttt{gsub\_map()} accepts a string and a mapping (named
list) of pattern-replacements, performing multiple \texttt{gsub()}
operations together:

\begin{Shaded}
\begin{Highlighting}[]
\NormalTok{gsub_map <-}\StringTok{ }\ControlFlowTok{function}\NormalTok{(s, mapping) \{}
    \CommentTok{# Accepts a mapping of pattern-replacements on a string s, allowing more }
    \CommentTok{# compact operations involving mutliple substitutions on the same string}
    \CommentTok{# sequentially}
    \ControlFlowTok{for}\NormalTok{ (i }\ControlFlowTok{in} \DecValTok{1}\OperatorTok{:}\KeywordTok{length}\NormalTok{(mapping)) \{}
\NormalTok{        pattern <-}\StringTok{ }\KeywordTok{names}\NormalTok{(mapping[i])}
\NormalTok{        replacement <-}\StringTok{ }\NormalTok{mapping[i]}
\NormalTok{        s <-}\StringTok{ }\KeywordTok{gsub}\NormalTok{(pattern, replacement, s)}
\NormalTok{    \}}
    \KeywordTok{return}\NormalTok{(s)}
\NormalTok{\}}

\CommentTok{# Example}
\KeywordTok{gsub_map}\NormalTok{(}\StringTok{'foo bar'}\NormalTok{, }\KeywordTok{list}\NormalTok{(}\DataTypeTok{foo=}\StringTok{'foo1'}\NormalTok{, }\DataTypeTok{bar=}\StringTok{'bar1'}\NormalTok{))}
\end{Highlighting}
\end{Shaded}

\begin{verbatim}
## [1] "foo1 bar1"
\end{verbatim}

Using \texttt{sapply()} to apply to function to each row of the columns:

\begin{Shaded}
\begin{Highlighting}[]
\NormalTok{df}\OperatorTok{$}\NormalTok{poisonous <-}\StringTok{ }\KeywordTok{sapply}\NormalTok{(df}\OperatorTok{$}\NormalTok{poisonous, gsub_map, }\KeywordTok{list}\NormalTok{(}\DataTypeTok{e=}\StringTok{'edible'}\NormalTok{, }\DataTypeTok{p=}\StringTok{'poisonous'}\NormalTok{))}
\NormalTok{df}\OperatorTok{$}\NormalTok{bruises <-}\StringTok{ }\KeywordTok{sapply}\NormalTok{(df}\OperatorTok{$}\NormalTok{bruises, gsub_map, }\KeywordTok{list}\NormalTok{(}\DataTypeTok{t=}\StringTok{'TRUE'}\NormalTok{, }\DataTypeTok{f=}\StringTok{'FALSE'}\NormalTok{))}
\NormalTok{df}\OperatorTok{$}\NormalTok{gill_size <-}\StringTok{ }\KeywordTok{sapply}\NormalTok{(df}\OperatorTok{$}\NormalTok{gill_size, gsub_map, }\KeywordTok{list}\NormalTok{(}\DataTypeTok{b=}\StringTok{'broad'}\NormalTok{, }\DataTypeTok{n=}\StringTok{'narrow'}\NormalTok{))}
\NormalTok{df}\OperatorTok{$}\NormalTok{ring_number <-}\StringTok{ }\KeywordTok{sapply}\NormalTok{(df}\OperatorTok{$}\NormalTok{ring_number, gsub_map, }\KeywordTok{list}\NormalTok{(}\DataTypeTok{n=}\StringTok{'0'}\NormalTok{, }\StringTok{'o'}\NormalTok{=}\StringTok{'1'}\NormalTok{, }\StringTok{'t'}\NormalTok{=}\StringTok{'2'}\NormalTok{))}

\KeywordTok{head}\NormalTok{(df)}
\end{Highlighting}
\end{Shaded}

\begin{verbatim}
##   poisonous bruises gill_size ring_number
## 1 poisonous    TRUE    narrow           1
## 2    edible    TRUE     broad           1
## 3    edible    TRUE     broad           1
## 4 poisonous    TRUE    narrow           1
## 5    edible   FALSE     broad           1
## 6    edible    TRUE     broad           1
\end{verbatim}

Finally, convert to proper R data types:

\begin{Shaded}
\begin{Highlighting}[]
\NormalTok{df}\OperatorTok{$}\NormalTok{bruises <-}\StringTok{ }\KeywordTok{as.logical}\NormalTok{(df}\OperatorTok{$}\NormalTok{bruises)}
\NormalTok{df}\OperatorTok{$}\NormalTok{ring_number <-}\StringTok{ }\KeywordTok{as.integer}\NormalTok{(df}\OperatorTok{$}\NormalTok{ring_number)}

\KeywordTok{head}\NormalTok{(df)}
\end{Highlighting}
\end{Shaded}

\begin{verbatim}
##   poisonous bruises gill_size ring_number
## 1 poisonous    TRUE    narrow           1
## 2    edible    TRUE     broad           1
## 3    edible    TRUE     broad           1
## 4 poisonous    TRUE    narrow           1
## 5    edible   FALSE     broad           1
## 6    edible    TRUE     broad           1
\end{verbatim}

\begin{Shaded}
\begin{Highlighting}[]
\KeywordTok{sapply}\NormalTok{(df, class)}
\end{Highlighting}
\end{Shaded}

\begin{verbatim}
##   poisonous     bruises   gill_size ring_number 
## "character"   "logical" "character"   "integer"
\end{verbatim}


\end{document}
